\documentclass[a4paper,11pt]{article}
\usepackage[utf8]{inputenc}
\usepackage[left=2cm,top=3cm,text={17cm, 24cm}]{geometry}
\usepackage[czech]{babel}
\usepackage{times}
\usepackage[T1]{fontenc}


\begin{document}

\begin{titlepage}
\begin{center}
\Huge\textsc{Vysoké učení technické v Brně} \\
\huge\textsc{Fakulta informačních technologií}
\vspace{\stretch{0.382}}
\\
\LARGE Typografie a publikování\,--\,4. projekt \\ \Huge{Bibliografické citace} \vspace{\stretch{0.618}}
\end{center}
{\Large \today \hfill
Jaroslav Kvasnička}
\end{titlepage}



\section{Historie Typografie}
Typografie jako taková se nejvíce začala rozvíjet po vynalezení knihtisku Johannem Gutenbergem. Typografie se dělí do několika oborů jako jsou makrotypografie a mikrotypografie viz \cite{HaleyAllan}.
\par
Makrotypografie se zabývá rozložením textu na stránku, jeho proporcemi, ilustracemi a celkovým vzhledem.
Zatímco mikrotypografie se zabývá detaily písma, jeho fonty, a uměleckou tvorbou písma \cite{MartinaVichova}.


\section{Pravidla typografie}
Od doby knihtisku po současnost se vyvinulo mnoho typografických pravidel. Základní pravidla jsou napřídklad optický střed, zlatý řez, normalizované formáty a další. Ikdyž se u moderních textů můžeme setkávat s narušením některých z nich, pravidlo čitelnosti by mělo vždy zůstat \cite{Vesely}. Při čtení textu by čtenář neměl být nikdy vyčerpávaný jeho nevzhledností \cite{JanRajlich}.


\section{Typografie v roce 2021}
V letošním roce vzniklo mnoho nových fontů. Zde je několik příkladů populárně používaných fontů, které se objevují v různých částí trhu, od webových tutoriálů přes design na oblečení a boty až po umění viz \cite{TheLogoCreative2021}:
\begin{itemize}
\item{Měnící základ}
\item{Pěvné stíny}
\item{Extra ostré úhly}
\item{Kineitcká typografie}
\item{Retro-futuristické fonty}
\item{Experimentální písmo}
\item{Písmo s animací}
\end{itemize}


\section{{\LaTeX}}
Tex je program používaný k psaní dokumentů, prezentací, vědeckých článků, kvalifikačních prací. \LaTeX je překladač, který se používá k přeložení zdrojového kódu do hotového textového souboru většinou s příponou \texttt{*.pdf} \cite{PavelBOJKO}. Existuje spousta programů, které používají \LaTeX jako například: {\TeX}maker, LyX, Overleaf, Papeeria a mnoho dalších viz \cite{BestLatexEditors}.


\section{{\TeX}}
Je program na úpravu textu, který kromě normálního sázení textu umí vysázet i matematické formule. {\TeX} také umí sázet písmena jiných než integrovaných jazyků jako například čeština, francouština \dots nicméně se musí použít příkaz \texttt{{\textbackslash}usepackage[language]\{babel\}}, kde language je jazyk, který chceme do {\TeX}u importovat \cite{PetrOlsak}.
\par
Také lze zapisovat a generovat matematické formule a rovnice. Ty se při překladu rozloží a sestaví se z abstraktní syntaktický strom. Nejprve se dělá podle relací až dojde k prvkům \cite{conferenc}.
\par
Příklad vysázené rovnice (lze nalést v~\cite{JiriKosek})
$$ \lim_{n \rightarrow \infty} \left(1 + {1 \over n} \right)^n = e $$


\newpage
\bibliographystyle{czechiso}
\bibliography{literature}


\end{document}