\documentclass[11pt,a4paper,twocolumn]{article}
\usepackage[utf8x]{inputenc}
\usepackage[czech]{babel}
\usepackage[left=1.5cm,text={18cm, 25cm},top=2.5cm]{geometry}
\usepackage[IL2]{fontenc}
\usepackage{times}
\usepackage{amsmath}
\usepackage{amsthm}
\usepackage{amsfonts}
\newtheorem{remark}{Věta}
\newtheorem{definition}{Definice}
\usepackage{setspace}


\begin{document}
\begin{titlepage}
\begin{center}
\Huge\textsc{Fakulta informačních technologií \\ Vysoké učení technické v Brně}\

\vspace{\stretch{0.382}}
\LARGE Typografie a publikování\,--\,2. projekt \linebreak Sazba dokumentů a matematických výrazů\
\vspace{\stretch{0.618}}
\end{center}
{\Large 2021 \hfill
Jaroslav Kvasnička (xkvasn14)}
\end{titlepage}




\section*{Úvod}
V této úloze si vyzkoušíme sazbu titulní strany, matematických vzorců, prostředí a dalších textových struktur obvyklých pro technicky zaměřené texty (například rovnice~(\ref{eq1})
nebo Definice \ref{def1} na straně \pageref{def1}). Rovněž si vyzkoušíme používání odkazů \verb|\ref| a \verb|\pageref|.\par
Na titulní straně je využito sázení nadpisu podle optického středu s využitím zlatého řezu. Tento postup byl
probírán na přednášce. Dále je použito odřádkování se
zadanou relativní velikostí 0.4\,em a 0.3\,em.\par
V případě, že budete potřebovat vyjádřit matematickou
konstrukci nebo symbol a nebude se Vám dařit jej nalézt
v samotném {\LaTeX}u, doporučuji prostudovat možnosti balíku maker \AmS-\LaTeX.


\section{Matematický text}
Nejprve se podíváme na sázení matematických symbolů a~výrazů v plynulém textu včetně sazby definic a vět s využitím balíku \verb|amsthm|. Rovněž použijeme poznámku pod
čarou s použitím příkazu \verb|\footnote|. Někdy je vhodné
použít konstrukci \verb|\mbox{}|, která říká, že text nemá být zalomen.

\begin{definition}\label{def1}
\textnormal{Rozšířený zásobníkový automat} (RZA) je definován jako sedmice tvaru $A = (Q, \Sigma, \Gamma, \delta, q_0, Z_0, F)$, kde:
\end{definition}

\renewcommand\labelitemi{$\bullet$}
\begin{itemize}
\item $Q$ je konečná množina vnitřních (řídicích) stavů,
\item $\Sigma$ je konečná vstupní abeceda,
\item $\Gamma$ je konečná zásobníková abeceda,
\item $\delta$ je přechodová funkce $Q \times (\Sigma \cup \{\epsilon\})\times\Gamma^* \to 2^{Q\times\Gamma^*}$,
\item $q_0 \in Q$ je počáteční stav,$Z_0 \in \Gamma$ je startovací symbol zásobníku a $F \subseteq Q$ je množina koncových stavů.
\end{itemize}\par
Nechť $P = (Q,\Sigma,\Gamma,\delta,q_0,Z_0,F)$ je rozšířený zásobníkový automat. \textit{Konfigurací} nazveme trojici $(q, w, \alpha) \in Q \times \Sigma^* \times \Gamma^*$, kde $q$ je aktuální stav vnitřního řízení, $w$ je dosud nezpracovaná část vstupního řetězce a $\alpha = Z_{i_{1}} Z_{i_{2}}\dotso Z_{i_{k}}$ je obsah zásobníku\footnote{$Z_{i_{1}}$ je vrchol zásobníku}.

\subsection{Podsekce obsahující větu a odkaz}

\begin{definition}\label{def2}
\textnormal{Řetězec $w$ nad abecedou $\Sigma$ je přijat RZA}\\A jestliže $(q_0,w,Z_0) \stackrel*{\underset{A}{\vdash}}  (q_F, \epsilon, \gamma)$ pro nějaké $\gamma \in \Gamma^*$ a $q_F \in F$. Množinu $L(A)$ = \{$w \mid w \text{ je přijat RZA A\}} \subseteq \\ \Sigma^*$ nazýváme \textnormal{jazyk přijímaný RZA} A.
\end{definition}

Nyní si vyzkoušíme sazbu vět a důkazů opět s použitím
balíku \texttt{amsthm}.

\begin{remark}\label{veta1}
Třída jazyků, které jsou přijímány ZA, \textnormal{odpovídá bezkontextovým jazykům.}

\end{remark}
\begin{proof}[Důkaz]\label{duk1}
V důkaze vyjdeme z Definice \ref{def1} a \ref{def2}.
\end{proof}

\section{Rovnice a odkazy}
Složitější matematické formulace sázíme mimo plynulý text. Lze umístit několik výrazů na jeden řádek, ale pak je třeba tyto vhodně oddělit, například příkazem \verb|\quad|.
\par

$$
\sqrt[i]{x_{i}^{3}}\quad \text { kde } x_{i} \text { je } i \text {-té sudé číslo splňující}\quad x_{i}^{x_{i}^{i^{2}}+2} \leq y_{i}^{x_{i}^{4}} $$

\par
V rovnici (1) jsou využity tři typy závorek s různou explicitně definovanou velikostí.


\begin{eqnarray}\label{eq1}
    x &=& \bigg[\Big\{\big[a+b\big] * c\Big\}^{d} \oplus 2\bigg]^{3 / 2}\\
    y &=& \lim _{x \rightarrow \infty} \frac{\frac{1}{\log _{10} x}}{\sin ^{2} x+\cos ^{2} x}\nonumber
\end{eqnarray}

%$$ y & = &\lim _{x \rightarrow \infty} \frac{\frac{1}{\log _{10} x}}{\sin %^{2} x+\cos ^{2} x} $$

V této větě vidíme, jak vypadá implicitní vysázení limity $\lim_{n\to\infty}f(n)$ v normálním odstavci textu. Podobně je to i s dalšími symboly jako $\prod_{i=1}^{n} 2^{i}$ či $\bigcap_{A \in \mathcal{B}}A$.~V~případě vzorců $\lim\limits_{n\to\infty}f(n)$ a $\prod\limits_{i=1}\limits^{n} 2^{i}$ jsme si vynutili méně úspornou sazbu příkazem \verb|\limits|.
\par
\begin{eqnarray}\label{eq2}
\int_{b}^{a} g(x) \mathrm{d}x &=& -\int\limits_{a}\limits^{b} f(x)  \mathrm{d}x
\end{eqnarray}


\section{Matice}
Pro sázení matic se velmi často používá prostředí \texttt{array} a závorky\,(\verb|\left|, \verb|\right|).


$$ \left(\begin{array}{ccc} a-b & \widehat{\xi+\omega} & \pi \\ \vec{\mathbf{a}} & \overleftrightarrow{A C} & \hat{\beta} \end{array}\right)=1 \Longleftrightarrow \mathcal{Q}=\mathbb{R} $$



$$ \mathbf{A}=\left\|\begin{array}{cccc} a_{11} & a_{12} & \ldots & a_{1 n} \\ a_{21} & a_{22} & \ldots & a_{2 n} \\ \vdots & \vdots & \ddots & \vdots \\ a_{m 1} & a_{m 2} & \ldots & a_{m n} \end{array}\right\|=\left|\begin{array}{rllr} t & u \\ v & w \end{array}\right|=tw \!-\! uv $$



Prostředí \texttt{array} lze úspěšně využít i jinde.

\begin{equation*}
 \binom{n}{k}\ = \Bigg\{
\begin{array}{cl}
0 & \text { pro } k<0 \text { nebo } k>n \\
\frac{n !}{k !(n-k) !} & \text { pro } 0 \leq k \leq n.
\end{array}
\end{equation*}


\end{document}
