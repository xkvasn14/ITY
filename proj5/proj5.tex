\documentclass[9pt]{beamer}
\usepackage[sfdefault]{roboto}
\usepackage[utf8]{inputenc}
\usepackage[T1]{fontenc}
\usepackage[czech]{babel}
\usepackage{styly/fluxmacros}
\usefolder{styly}
\usetheme[style=asphalt]{flux}
\usepackage{listings}
\usepackage{booktabs}
\usepackage{graphics}
\usepackage{colortbl}
\usepackage{ragged2e}
\usepackage{schemabloc}
\usepackage{hyperref}
\usebackgroundtemplate{
\includegraphics[width=\paperwidth,height=\paperheight]{pictures/grey-abs.jpg}}
\usepackage[czech]{babel}



\title{SPOJENÉ SEZNAMY}
\subtitle{JEDNOSMĚRNÝ, DVOJSMĚRNÝ}

\author{JAROSLAV KVASNIČKA}
\institute{VYSOKÉ UČENÍ TECHNICKÉ V BRNĚ \\FAKULTA INFORMAČNÍCH TECHNOLOGIÍ}
\titlegraphic{pictures/chain-logo.png}



\begin{document}
\titlepage


\begin{frame}
\frametitle{TABLE OF CONTENTS}
\tableofcontents
\end{frame}


\section{TROCHA TEORIE}
\begin{frame}{TROCHA TEORIE}
    \begin{itemize}
        \item Lineární datová struktura
        \item Prvky na sebe ukazují pomocí ukazatelů
        \item Každý prvek obsahuje data a ukazatele na následující, nebo přechozí prvek
        \item Pokud hlavička (začátek seznamu) ukazuje na patičku (konec seznamu), pak je seznam prázdný
    \end{itemize}
    \begin{itemize}
        \setbeamercolor*{itemize item}{fg=green}
        \item Velikost obsažené paměti je přímo závislá na počtu prvků
        \item rychlost při vkládání a odebáríná je vždy stejně vysoká
    \end{itemize}
    \begin{itemize}
        \setbeamercolor*{itemize item}{fg=red}
        \item Pomalé vyhledávání prvku \textit{i}
        \item Pouze sekvenční přístup k prvkům
    \end{itemize}
\end{frame}


\section{JEDNOSMĚRNÝ SEZNAM}
\begin{frame}[fragile]{JEDNOSMĚRNÝ SEZNAM}
\begin{figure}[h]
\scalebox{0.4}{\includegraphics{pictures/linked_list.png}}
\end{figure}
\begin{itemize}
    \item Struktura prvku obsahuje pouze data a ukazatel na další prvek
    \item Hledání prvku
    \begin{lstlisting}
bool search(struct Node* head, int x){
    struct Node* current = head;
    while (current != NULL){
        if (current->key == x)
            return true;
        current = current->next;
    }
    return false;
}
    \end{lstlisting}
\end{itemize}
\end{frame}


\section{ZÁKLADNÍ OPERACE JEDNOSMĚRNÉHO SEZNAMU}
\begin{frame}[fragile]{ZÁKLADNÍ OPERACE JEDNOSMĚRNÉHO SEZNAMU}
\begin{itemize}
\item Odstranění prvku ze seznamu
    \begin{lstlisting}[language=C]
void deleteNode(Node** head, int key)
{
    Node* temp = *head;
    Node* prev = NULL;
    if (temp != NULL && temp->data == key){
        *head = temp->next;
        delete temp;            
        return;
    }
    else{
    while (temp != NULL && temp->data != key){
        prev = temp;
        temp = temp->next;
    }
    if (temp == NULL)
        return;
    prev->next = temp->next;
    delete temp;
    }
}
    \end{lstlisting}
\end{itemize}
\end{frame}


\begin{frame}[fragile]{ZÁKLADNÍ OPERACE JEDNOSMĚRNÉHO SEZNAMU}
\begin{itemize}
\item Vložení prvku nakonec seznamu
    \begin{lstlisting}[language=C]
void append(struct Node** head, int new_data)
{
    struct Node* new_node =
(struct Node*) malloc(sizeof(struct Node));
    struct Node *last = *head;
    new_node->data  = new_data;
    new_node->next = NULL;
    if (*head == NULL){
       *head = new_node;
       return;
    }  
    while (last->next != NULL)
        last = last->next;
    last->next = new_node;
    return;    
}
    \end{lstlisting}
\end{itemize}
\end{frame}


\section{DVOJSMĚRNÝ SEZNAM}
\begin{frame}[fragile]{DVOJSMĚRNÝ SEZNAM}
\begin{figure}[h]
\scalebox{0.4}{\includegraphics{pictures/doub_linked_list.png}}
\end{figure}
\begin{itemize}
    \item Struktura obsahuje data a ukazatele na další a předchozí prvek
    \item Hledání v seznamu
    \begin{lstlisting}
int search(Node** head, int x)
{
    Node* temp = *head;
    int pos = 0;
    while (temp->data != x && temp->next != NULL) {
        pos++;
        temp = temp->next;
    }
    if (temp->data != x)
        return -1;
    return (pos + 1);
}
    \end{lstlisting}
\end{itemize}
\end{frame}


\section{ZÁKLADNÍ OPERACE OBOUSMĚRNÉHO SEZNAMU}
\begin{frame}[fragile]{ZÁKLADNÍ OPERACE OBOUSMĚRNÉHO SEZNAMU}
\begin{itemize}
\item Odstranění prvku ze seznamu
    \begin{lstlisting}[language=C]
void deleteNode(struct Node** head_ref, struct Node* del){
    if (*head_ref == NULL || del == NULL)
        return;
    if (*head_ref == del)
        *head_ref = del->next;
    if (del->next != NULL)
        del->next->prev = del->prev;
    if (del->prev != NULL)
        del->prev->next = del->next;
    free(del);
    return;
}
    \end{lstlisting}
\end{itemize}
\end{frame}


\begin{frame}[fragile]{ZÁKLADNÍ OPERACE OBOUSMĚRNÉHO SEZNAMU}
\begin{itemize}
\item Vložení prvku do seznamu za daný prvek
    \begin{lstlisting}[language=C]
void insertAfter(struct Node* prev_node, int new_data){
    if (prev_node == NULL) {
        printf("the given previous node cannot be NULL");
        return;
    }
    struct Node* new_node = (struct Node*)malloc(sizeof(struct Node));
    new_node->data = new_data;
    new_node->next = prev_node->next;
    prev_node->next = new_node;
    new_node->prev = prev_node;
    if (new_node->next != NULL)
        new_node->next->prev = new_node;
}
    \end{lstlisting}
\end{itemize}
\end{frame}




\begin{frame}{POUŽITĚ ZDROJE}
\begin{thebibliography}{}
\setbeamertemplate{bibliography item}[online]
\bibitem[Geeksforgeeks]{} Geeksforgeeks
\newblock \texttt{https://www.geeksforgeeks.org/}
\bibitem[Overleaf]{} Template
\newblock \texttt{https://cs.overleaf.com/latex/templates/}
\bibitem[Overleaf]{} Beamer theory
\newblock \texttt{https://cs.overleaf.com/learn/latex/beamer}
\end{thebibliography}
\end{frame}


\end{document}
